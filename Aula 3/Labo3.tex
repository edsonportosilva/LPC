\documentclass[12pt,addpoints]{exam}
\usepackage[utf8]{inputenc}
\usepackage[T1]{fontenc}
\usepackage[brazil]{babel}
\usepackage[a4paper, margin=2cm]{geometry}
\usepackage{graphicx, amsmath, amsfonts, amssymb, xcolor, url, tikz, pgfplots, subfigure}
\usepackage{gensymb}

\newcommand{\disciplina}{Laboratório de Princípios de Comunicações}
\newcommand{\periodo}{2023.1}
\newcommand{\avaliacao}{Guia de Experimentos 3}
\newcommand{\tema}{Modulação e Desmodulação em Amplitude}
%\newcommand{\professor}{Bruno\ B.\ Albert e Edmar C.\ Gurjão}
%\newcommand{\professor}{Leocarlos B.\ S.\ Lima e Edson P.\ da Silva}
\newcommand{\professor}{Edson P.\ da Silva e Luciana Veloso}
%\newcommand{\professor}{Edmar C.\ Gurjão e Luciana Veloso}
%\newcommand{\professor}{Bruno\ B.\ Albert e Edson P.\ da Silva}
%\newcommand{\professor}{Adolfo Herbster e Bruno Albert}
%\newcommand{\professor}{Edson Porto e Bruno Albert}
%\newcommand{\professor}{Bruno Albert e Luciana Veloso}

\pagestyle{head}
\firstpageheader{}{}{}
\runningheader{Lab.\ Princ.\ Comunicações}{\avaliacao}{Página \thepage}
\runningheadrule
\pointpoints{ponto}{pontos}
\newcommand{\myscale}{0.4}
\newtheorem{exemplo}{Exemplo}[section]

\begin{document}
    
\noindent \includegraphics[height=2cm]{../Figuras/UFCGLogo.png} \hfill
\begin{minipage}{.66\textwidth} \large \centering \vspace{-1.8cm}
    Universidade Federal de Campina Grande -- UFCG \\
    Unidade Acadêmica de Engenharia Elétrica -- UAEE \\
    Curso de Graduação em Engenharia Elétrica
\end{minipage}
\hfill \includegraphics[height=2cm]{../Figuras/DEELogo.png} \\[12pt]

\noindent
\begin{tabular*}{\textwidth}{l @{\extracolsep{\fill}} r @{\extracolsep{6pt}} l}
    \textbf{\disciplina} && \\
    Período \periodo && \\
    \textbf{\avaliacao} && \\
    Tema(s): \tema && \\
    Professor(es): \professor && \\
\end{tabular*}
\noindent\rule[2ex]{\textwidth}{2pt}
    
\section{Introdução}

O presente guia descreve atividades experimentais a serem realizadas na disciplina Laboratório de Princípios de Comunicações do curso de graduação em Engenharia Elétrica da Universidade Federal de Campina Grande -- UFCG.

Os experimentos propostos deverão ser realizados no Laboratório de Princípios de Comunicações -- LPC, localizado na Central de Laboratórios da Unidade Acadêmica de Engenharia Elétrica da UFCG, empregando:
\begin{itemize}
    \item Computador com software GNU Radio Companion -- GRC (\url{http://gnuradio.org/}) instalado;
    \item Módulo USRP (do inglês \textit{Universal Software Radio Peripheral}) para transmissão e recepção de sinais numa abordagem conhecida como Rádio Definido por Software -- RDS.
\end{itemize}

Na seção \ref{sect:Preparacao} deste guia, é sugerido um conjunto de atividades de preparação que devem ser concluídas antes da aula em que as práticas experimentais serão conduzidas. A realização prévia da preparação é fundamental para o bom desenvolvimento das práticas, tanto em termos de tempo quanto de aprendizado. Portanto, \textbf{cada discente só poderá realizar os experimentos em laboratório após apresentarem ao professor responsável, no início da aula, os resultados da preparação proposta}. 


A aula terá duração de duas horas e cada discente deverá entregar ao seu término, por escrito, respostas às questões referentes aos experimentos realizados propostas na Folha de Respostas (parte final do guia).

\section{Objetivos}

As práticas experimentais aqui propostas têm por objetivos:
\begin{itemize}
    \item Simular e analisar a modulação em amplitude;
    \item Simular e analisar o desmodulador síncrono;
    \item Simular e analisar o desmodulador por detecção de envoltória;
    \item Investigar o conceito de um receptor super-heteródino.
\end{itemize}

\section{Preparação} \label{sect:Preparacao}

\subsection{Estudo}

Revise e pesquise sobre os conceitos:
\begin{itemize}
    \item AM-DSB-SC e AM-DSB;
    \item Efeitos da incoerência de fase e de frequência na desmodulação síncrona no AM-DSB-SC e AM-DSB;
    \item Misturador (\textit{mixer}) e receptor super-heteródino.
\end{itemize}

\subsection{Problemas}

Os problemas propostos a seguir devem ser obrigatoriamente resolvidos e colocados na plataforma Moodle para se ter acesso ao relatório do experimento. % apresentados por escrito ao professor antes do início das práticas de laboratório.
Os resultados destes problemas serão necessários para a realização dos experimentos propostos. 

\begin{questions}
  \question Considere um tom modulante de frequência $f_{m} = 200$~Hz e uma portadora de frequência $f_{c} = 5$~kHz. Esboce os gráficos, no tempo e na frequência, do sinal modulante $m(t) = \cos{2\pi f_{m}t}$ e do sinal modulado $\varphi_{DSB}(t) = A\cos{2\pi f_{c}t} + m(t)\cos{2\pi f_{c}t}$ considerando índices de modulação $\mu = 0.5$, $1.0$, $2.0$ e $\infty$. %($A = 0$).
  \question Descreva um diagrama de blocos de um receptor síncrono (coerente) para $\varphi_{DSB}(t)$.
  \question Descreva um diagrama de blocos de um detector de envoltória para $\varphi_{DSB}(t)$.
  %\question O AM comercial usa um receptor super-heteródino com detecção de envoltória. Nele, o sinal recebido é deslocado para uma frequência intermediária $f_{i} = 455$~kHz. Quais as frequências do oscilador local do misturador para $f_{c} = 1050$, $1160$ e $1310$~kHz para deslocar o sinal para a frequência intermediária $f_i$?
\end{questions}

\section{Experimentos}

A seguir são descritas práticas experimentais a serem realizadas pelo aluno em aula de laboratório. 

\subsection{Experimento 1 -- Modulação em Amplitude}

O objetivo deste experimento é analisar as modulações em amplitude DSB e DSB-SC.

\begin{enumerate}   
    \item Execute o software GRC e abra o arquivo \textbf{Labo3-1.grc}. A Figura \ref{fig:GRC_3-1} ilustra o diagrama deste experimento. Ele consiste na simulação do sinal AM-DSB $\varphi_{DSB}(t)$, dado por 
\begin{align} \label{eq:am}
    \varphi_{DSB}(t) =& [A + m(t)]\cos(2\pi f_{c}t),\\
        		    =& A\cos(2\pi f_{c}t) + m(t)\cos(2\pi f_{c}t)    
\end{align}
em que $m(t)$ é o sinal modulante (com nível DC zero), $A$ corresponde à amplitude da portadora não-modulada e $f_c$ é a frequência da portadora em Hz. Nesta simulação, utilizaremos como sinal modulante $m(t)$ um sinal senoidal, o qual facilmente podemos reproduzir e variar sua amplitude e frequência. Os parâmetros $A$ e $f_c$ podem ser alterados por suas respectivas réguas deslizantes disponíveis acima dos gráficos durante a execução do diagrama.
    \begin{figure}[htb]
        \centering
        \includegraphics[scale=\myscale]{./Figuras/Labo3-1}
        \caption{Diagrama de blocos para análise da modulação em amplitude.} 
        \label{fig:GRC_3-1}
    \end{figure}
  \item Execute o diagrama e responda às questões propostas na Folha de Respostas.
\end{enumerate}

\subsection{Experimento 2 -- Desmodulação Síncrona}

O objetivo deste experimento é observar o efeito do desvio de fase e do desvio de frequência num receptor coerente.

\begin{enumerate}
    \item  Abra o arquivo \textbf{Labo3-2.grc} disponibilizado pelo professor. A Figura \ref{fig:GRC_3-1b} ilustra o diagrama deste experimento. Ele consiste do modulador visto no Experimento 1 e de um desmodulador síncrono (coerente), ou seja, o oscilador do receptor tem a mesma frequência e a mesma fase da portadora gerada no transmissor. 
    \begin{figure}[htb]
        \centering
        \includegraphics[scale=\myscale]{./Figuras/Labo3-2}
        \caption{Diagrama de blocos de um desmodulador síncrono.} 
        \label{fig:GRC_3-1b}
    \end{figure}
  \item Execute o experimento e observe que o sinal desmodulado tem metade da amplitude do sinal transmitido.
  % \item  Altere o índice de modulação e observe que a desmodulação coerente desmodula o sinal modulado para qualquer índice de modulação do transmissor.
  \item Responda as questões propostas na Folha de Respostas.
\end{enumerate}

\subsection{Experimento 3 -- Desmodulação por Detecção da Envoltória}

O objetivo deste experimento é observar o efeito da sobremodulação ($\mu > 1$) na desmodulação por deteção da envoltória (não coerente).

\begin{enumerate}
    \item  Abra o arquivo \textbf{Labo3-3.grc} disponibilizado pelo professor. A Figura \ref{fig:GRC_3-1c} ilustra o diagrama deste experimento. Ele consiste do modulador visto no Experimento 1 e de um desmodulador por detecção da envoltória que consiste de um retificador seguido de um filtro passa-baixas.
    \begin{figure}[htb]
        \centering
        \includegraphics[scale=\myscale]{./Figuras/Labo3-3}
        \caption{Diagrama de blocos de um desmodulador por detecção da envoltória.} 
        \label{fig:GRC_3-1c}
    \end{figure}
  \item Responda as questões propostas na Folha de Respostas.
\end{enumerate}

%\subsection{Experimento 4 -- Receptor Super-heteródino}
%
%O objetivo deste experimento é mostrar o conceito de um receptor super-heteródino.
%
%\begin{enumerate}
%    \item  Abra o arquivo \textbf{Labo3-4.grc} disponibilizado pelo professor. A Figura \ref{fig:GRC_3-1d} ilustra o diagrama deste experimento. Ele consiste de um modulador AM-DSB para um sinal de voz gravado. No esquema, é possível alterar as frequências da portadora e de sintonia do receptor bem como a amplitude da portadora. A frequência intermediária é $fi = 25$~kHz, de modo que toda vez que uma frequência é sintonizada o sinal é transladado para $fi$. 
%    \begin{figure}[htb]
%        \centering
%        \includegraphics[scale=\myscale]{./Figuras/Labo3-4}
%        \caption{Diagrama de blocos de um receptor super-heteródino.} 
%        \label{fig:GRC_3-1d}
%    \end{figure}
%  \item Execute o experimento e responda as questões propostas na Folha de Respostas.
%\end{enumerate}

\subsection{Experimento 4 -- Modulação AM-SSB}

O objetivo deste experimento é simular um sistema de transmissão AM-SSB com detecção síncrona e compará-lo ao sistema AM-DSB.

\begin{enumerate}
    \item  Abra o arquivo \textbf{Labo3-5.grc} disponibilizado pelo professor. A Figura \ref{fig:GRC_3-5} ilustra o diagrama deste experimento. Neste diagrama, um sinal AM-SSB é gerado com o auxílio de um filtro de Hilbert no transmissor.
    \begin{figure}[htb]
        \centering
        \includegraphics[scale=\myscale]{./Figuras/Labo3-5}
        \caption{Diagrama de blocos de um sistema AM-SSB com detecção síncrona.} 
        \label{fig:GRC_3-5}
    \end{figure}
  \item Responda as questões propostas na Folha de Respostas.
\end{enumerate}

\newpage \pagenumbering{arabic}

\noindent \includegraphics[height=2cm]{../Figuras/UFCGLogo} \hfill
\begin{minipage}{.66\textwidth} \large \centering \vspace{-1.8cm}
    Universidade Federal de Campina Grande -- UFCG \\
    Unidade Acadêmica de Engenharia Elétrica -- UAEE \\
    Curso de Graduação em Engenharia Elétrica
\end{minipage}
\hfill \includegraphics[height=2cm]{../Figuras/DEELogo} \\[12pt]

\noindent
\begin{tabular*}{\textwidth}{l @{\extracolsep{\fill}} r @{\extracolsep{6pt}} l}
    \textbf{\disciplina} && \\
    Período \periodo && \\
    \textbf{\avaliacao\ -- Folha de Respostas} && \\
    Tema(s): \tema && \\
    Professor(es): \professor && \\[12pt]
    \textbf{Aluno:} \hrulefill & \textbf{Data:} \makebox[3cm]{\hrulefill} & \\
\end{tabular*}
\noindent\rule[2ex]{\textwidth}{2pt}

\section*{Experimento 1 -- Modulação em Amplitude}

\begin{questions}
    \question Para $A = 0$ temos um transmissor AM-DSB-SC, esboce o gráfico no domínio da frequência obtido e identifique o sinal modulante e o sinal modulado com suas bandas laterais. Apresente todos os valores pertinentes, frequências e amplitudes, com suas unidades no gráfico. Qual o índice de modulação nesse caso?
    \makeemptybox{5cm}
    
    % \question Faça $A = 1$ (deslizando a respectiva régua) observe no espectro de frequência a introdução do espectro da portadora  em relação a questão anterior, quando $A=0$. Qual o novo índice de modulação? Qual a potência média da portadora? Qual a amplitude da portadora em dB no espectro de frequência?
    % \fillwithlines{0.25in}
    \question Qual a largura de faixa do sinal modulante e do sinal modulado?
    \fillwithlines{0.5in}

    \question Alterar a frequência da portadora, para 3 kHz por exemplo, altera a largura de faixa do sinal modulado? Altera a sua potência? Justifique.
    \fillwithlines{0.75in}

    \question Faça $A = 1$ (deslizando a respectiva régua) observe no espectro de frequência a introdução do espectro da portadora  em relação a questão anterior, quando $A=0$. Qual é o valor do índice de modulação? Observe que a envoltória do sinal modulado segue o sinal mensagem e que a introdução da portadora não alterou a largura de faixa do sinal modulado. % Qual a diferença entre a potência da portadora e da mensagem em dB observando o espectro de frequência? Quantas vezes a potência da portadora é maior do que a do sinal mensagem?
    \fillwithlines{0.5in}

%    \question A introdução da portadora alterou a largura de faixa do sinal modulado?
%    \fillwithlines{0.25in}

    \question Quanto de potência a mais temos que ter no transmissor para transmitir o sinal modulado com $A=1$ em relação ao sinal modulado com $A=0$?
    \fillwithlines{0.5in}

    % \question Altere a frequência do sinal modulante para 400 Hz. Que alteração foi observada no espectro de frequência do sinal modulado?
    % \fillwithlines{0.25in}
    
    % \question Faça $A = 3$ observe o espectro de frequência. Qual o novo índice de modulação? Qual a potência média da portadora? Qual a amplitude da portadora em dB no espectro de frequência?
    % \fillwithlines{0.25in}
    
    % \question Faça $A = 2$ observe o espectro de frequência. Qual o novo índice de modulação?  Qual a diferença entre a potência da portadora e da mensagem em dB observando o espectro de frequência? Quantas vezes a potência da portadora é maior do que a do sinal mensagem?
    % \fillwithlines{0.25in}

    % \question Qual o esquema é o mais eficiente, e o menos eficiente em termos de potência do transmissor, $A=0$, $A = 1$ ou $A = 2$? Justifique.
    % \fillwithlines{0.75in}
    
    % \question Altere a frequência da portadora deslizando a sua respectiva régua. O que aconteceu com o sinal modulado? (Observe o gráfico em frequência.)
    % \fillwithlines{0.5in}
    
    %\question Coloque uma onda quadrada como mensagem e ajuste o parâmetro \textbf{Offset} para $-0.5$. Execute experimento para $A = 0$ e verifique que não há mais diferenças nas amplitudes do sinal. Explique onde está a diferença entre os níveis da onda quadrada? Esse fato não ocorre para $A \ne 0$. Nota: Essa é uma modulação digital chamada de PSK ({\it Phase Shift Keing}).
   % \fillwithlines{0.5in}
    
 

\end{questions}

\section*{Experimento 2 -- Desmodulação Síncrona}

\begin{questions}
  
    % \question  Por que são necessários o filtro passa-baixas e o bloqueador DC no desmodulador? Como sugestão, retire-os um de cada vez e observe o que ocorre.
    % \fillwithlines{1in}

  
    \question Altere o índice de modulação deslizando a régua da amplitude da portadora e observe que a desmodulação coerente desmodula o sinal modulado para qualquer valor do índice de modulação do transmissor. Por que?
    \fillwithlines{0.5in}

   \question Retire o filtro passa-baixas, refaça a conexão e explique sua utilidade observando os gráficos na frequência e no tempo
   \fillwithlines{0.5in}

   \question Retorne o filtro (use {\bf Ctrl-Z}). Retire agora o bloqueador DC refaça as conexões e explique sua utilidade quando:
   \begin{parts}
     \part A amplitude da portadora, régua deslizante, for zero e
     \part A amplitude da portadora, régua deslizante, for um.
   \end{parts}
   \fillwithlines{1in}
   
%    \question Retorne o filtro ({\bf Ctrl-Z}). A fase do oscilador local no receptor pode ser alterada pelo bloco \textbf{Delay} que atrasa o sinal num tempo correspondente a um certo número de amostras. Qual é esse tempo para um atraso de uma amostra? Assim, cada amostra corresponde a um atraso de $1/f_{s} = 1/25000 =  40$~$\mu$s, em que $f_{s}$ é a frequência de amostragem, ou a uma defasagem de $\frac{1/f_{m}}{1/f_{s}}.360º = \frac{1/5000}{1/25000}.360º = 72º$ no oscilador local. Faça esse atraso igual a 1 amostra. Houve distorção no sinal desmodulado em relação ao sinal mensagem? Explique.
%    \fillwithlines{0.75in}

    \question Retorne o bloqueador DC. A fase do oscilador local no receptor pode ser alterada pelo bloco \textbf{Delay} que atrasa o sinal num tempo correspondente a um certo número de amostras. Cada amostra corresponde a um atraso de $1/f_{s} = 1/25000 =  40$~$\mu$s, em que $f_{s}$ é a frequência de amostragem, ou a uma defasagem de $\frac{f_{ol}}{f_{s}}.360\degree = \frac{5000}{25000}.360\degree = 72\degree$ no oscilador local, em que $f_{ol}$ é a frequência do oscilador local. Deslize a régua da fase para $72\degree$. Houve distorção no sinal desmodulado em relação ao sinal mensagem? Explique.
    \fillwithlines{0.5in}
    
    \question Retorne a régua da fase para 0\degree. Altere o oscilador local do receptor deslizando a régua da frequência do receptor (Freq. receptor) para 5010~Hz e execute o experimento. Por que o sinal desmodulado está distorcido em relação ao sinal mensagem? Seria aceitável 0,1~Hz de diferença (oscilador local em 5000,1~Hz)?
    \fillwithlines{0.75in}
\end{questions}

\section*{Experimento 3 -- Desmodulação por Detecção da Envoltória}

\begin{questions}
    \question Para amplitudes da portadora em 0.0, 0.5, 1.0 e 2.0 (use  a régua deslizante para isso), observe os resultados (sinais desmodulados) no tempo e na frequência para cada valor. Quais resultados correspondem ao sinal modulante (mensagem)? Por que?
    \fillwithlines{0.75in}
\end{questions}

%\section*{Experimento 4 -- Receptor Super-heteródino}
%
% \begin{questions}
%%     \question Por que o sinal sintonizado é sempre transladado para $f_{i} = 25$~kHz? Observe o bloco \textbf{Signal Source} correspondente ao oscilador local do receptor e use os resultados da preparação.
%%     \fillwithlines{1in}
%    
%    \question Na prática é possível se estimar o índice de modulação pela observação da envoltória do sinal modulado. Considere $S_{max}$ e $S_{min}$ os valores máximo e mínimo da envoltória respectivamente, então
%    \begin{equation}
%      \label{eq:mu100}
%      \mu \times 100\% = \frac{S_{max} - S_{min}}{S_{max} + S_{min}}
%      \times 100\%
%    \end{equation}
%    Estime o índice de modulação do sinal de voz modulado desse experimento. Para qual valor de amplitude teríamos um indice de modulação de aproximadamente 1? Altere o índice de modulaçao para infinito ($A = 0$) e observe o que ocorre.
%    \fillwithlines{0.5in}
%
%   \question Coloque o transmissor AM (a frequência da  portadora do transmissor) na frequência de 125~kHz. Sintonize na nova frequência, perceba que não precisa termos a frequência exata para que o sinal de voz seja capturado. Existe uma outra frequência de sintonia no receptor, além da de 125~kHz, em que o sinal é também desmodulado. A rádio em 125~kHz é uma rádio fantasma para essa frequência. Qual é essa frequência e por que isso ocorre?
%   \fillwithlines{1in}
%
% \end{questions}

\section*{Experimento 4 -- Modulação AM-SSB}

\begin{questions}
    \question Quais são as diferenças entre os sinais AM-SSB e AM-DSB no domínio do tempo? E no domínio da frequência?
    \fillwithlines{1.5in}
    
    \question De que outra maneira o sinal AM-SSB poderia ser gerado a partir do sinal AM-DSB?
    \fillwithlines{1.5in}
    
\end{questions}


\end{document}
